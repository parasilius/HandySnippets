\documentclass{article}
\usepackage{amsmath}

\author{DrNakiz::HandySnippets}
\title{Sum of All Proper Divisors of a Natural Number}

\begin{document}

\maketitle

\section{Proofs}

\subsection{First Algorithm}

The first algorithm is initialized with a list including an item of 1.

We loop through the numbers in the range $[2, \frac{n}{2}]$, in which $n$ is
our natural number. For each integer $i$ in the mentioned range, we check if
$n$ and $\frac{n}{i}$ is not already in our list. If not, we add it. Finally, 
the sum of all items in the list is returned.

Note that this is not an optimized algorithm, but easier to understand. The 
reason we check up to $\frac{n}{2}$ is that 2 is the first prime number that 
may or may not be a proper divisor of the natural number $n$, but in each case,
 all the other proper divisors are less than or equal to $\frac{n}{2}$.


\subsection{Second Algorithm}

The second algorithm was published in here\footnote{https://www.geeksforgeeks.org/sum-of-all-proper-divisors-of-a-natural-number/\#:~:text=Given\%20a\%20natural\%20number\%2C\%20calculate,\%2B\%205\%20\%2B\%2010\%20\%3D\%2022.},
but I found the explanation irrelevant. I found a better explanation here\footnote{https://stackoverflow.com/questions/5811151/why-do-we-check-up-to-the-square-root-of-a-prime-number-to-determine-if-it-is-pr}:

If a natural number $n$ is not a prime, it can be factored into two factors 
$a$ and $b$.

Now $a$ and $b$ can't be both greater than $\sqrt{n}$, because if $\sqrt{n} < a$
and $\sqrt{n} < b$, then as far as $n$, $a$ and $b$ are positive numbers, 
we will obtain $n < ab$. Or a more complete proof is:

\begin{align*}
    \sqrt{n} &< a
\end{align*}
Multiplying both sides by $b$ we have:
\begin{align*}
    \sqrt{n}\times b &< a\times b
\end{align*}
Dividing both sides by $\sqrt{n}$ we obtain:
\begin{align*}
    b &< \frac{a\times b}{\sqrt{n}}
\end{align*}
Now by having $\sqrt{n} < b$ and $b < \frac{a\times b}{\sqrt{n}}$, we obtain:
\begin{align*}
    \sqrt{n} &< \frac{a\times b}{\sqrt{n}}
\end{align*}
Multiplying both sides by $\sqrt{n}$ we have:
\begin{align*}
    n &< ab
\end{align*}
Which is false. So if one factor is lesser than $\sqrt{n}$, the other one is
greater than $\sqrt{n}$. Also note that if $\sqrt{n}$ is a natural number
itself, we have to include it.

\end{document}
